\documentclass[11pt,english,
               BCOR=10mm,DIV=10,% <- changed
               bibliography=totoc,parskip=half*,<- changed
               headings=small,% <- changed
               headinclude=false,% <- changed
               footinclude=false,oneside,mpinclude=false]{pst-doc}
\AfterCalculatingTypearea{%
  \newgeometry{top=0.5in, bottom=0.3in,left=1.4in,right=0.7in,%
               marginparwidth=1.3in,marginparsep=0.1in,footskip=0.2in,%
               headheight=1cm,headsep=0.27cm,reversemarginpar}%
  \KOMAoptions{headwidth=text,headsepline=:text,footwidth=text}% <- added
}
\recalctypearea
\RedeclareSectionCommand[
  %runin=false,
  afterindent=false,
  beforeskip=\baselineskip,
  afterskip=.5\baselineskip]{section}
\RedeclareSectionCommand[
  %runin=false,
  afterindent=false,
  beforeskip=.75\baselineskip,
  afterskip=.5\baselineskip]{subsection}
\RedeclareSectionCommand[
  %runin=false,
  afterindent=false,
  beforeskip=.5\baselineskip,
  afterskip=.25\baselineskip]{subsubsection}
\listfiles
\let\Lfile\LFile
\usepackage{biblatex}
\addbibresource{\jobname.bib}
%\lstset{language=PSTricks,basicstyle=\footnotesize\ttfamily}
\def\DVI{\textsc{DVI}}
\def\PDF{\textsc{PDF}}
\def\gs{\textsc{Ghostscript}}
\usepackage{multicol,enumitem,hologo,accsupp,csquotes}

% Identification
\def\myscript{pst2pdf}
\def\fileversion{0.19}
\def\filedate{2020/08/19}

% Config hyperref
\hypersetup{
  linktoc            = all,
  pdftitle           = {.:: pst2pdf v\fileversion{0.19} [\filedate] --- Running a PSTricks document with pdflatex ::.},%
  pdfauthor          = {Pablo Gonz\'{a}lez Luengo},%
  pdfsubject         = {Documentation for version \fileversion},%
  pdfkeywords        = {extract, conversion, images, pstricks, standalone files},
  pdfstartview       = {FitH},%
  bookmarksopenlevel = 2,%
}

% Some font for verbatim examples
\newfontfamily\lmmitalic{lmmono10-italic.otf}[
   Scale             = 0.95,%
   Extension         = .otf,%
   ItalicFont        = lmmono10-italic,%
   SmallCapsFont     = lmmonocaps10-oblique,%
   SlantedFont       = lmmonoslant10-regular,
   ]
% The character of visible space is now taken from Latin Modern Mono
% to prevent fonts in T1. The original definition for xetex/luatex is
% \def\verbvisiblespace{\usefont{OT1}{cmtt}{m}{n}\asciispace}
\def\verbvisiblespace{{\fontfamily{lmtt}\selectfont\char"2423}}
% Add some colors
\definecolor{optcolor}{rgb}{0.281,0.275,0.485}
\definecolor{pkgcolor}{rgb}{0,0.5,0.5}
\definecolor{araracolor}{rgb}{0, 0.72, 0.28}
\definecolor{linkcolor}{rgb}{0.04,0.38,0.04}
\definecolor{rulecolor}{rgb}{0.96,0.96,0.96}

% new macros
\makeatletter
\def\Lscript#1{\texttt{#1}\index{#1@\texttt{#1}}\index{Script!#1@\texttt{#1}}}
\def\Losystem#1{\texttt{#1}\index{#1@\texttt{#1}}\index{Operating system!#1@\texttt{#1}}}
\def\Loptcmd#1{\index{Options in command line!#1@\texttt{#1}}}% \index{#1@\texttt{#1}}
\def\Limgform#1{\texttt{#1}\index{Image format!#1@\texttt{#1}}}% \index{#1@\texttt{#1}}
\def\Loptprg#1{\index{Compiler options!#1@\texttt{#1}}}% \index{#1@\texttt{#1}}
\def\Venv#1{\texttt{#1}\index{Environment Verbatim!#1@\texttt{#1}}}
\def\Wenv#1{\texttt{#1}\index{Environment Verbatim Write!#1@\texttt{#1}}}

% Only index a starred version
\RenewDocumentCommand{\LFile}{sm}
  {
    \IfBooleanTF{#1}%
      {%
        \LFile{#2}

      }%
      {%
        \texttt{#2}%
      }%
  }%

% \prgopt{sm} : #1 compiler opt, #2 program opt:
\NewDocumentCommand{\prgopt}{sm}
  {
    \IfBooleanTF{#1}%
      {%
        \mbox{\textcolor{gray}{\texttt{-{}#2}}}%
        \Loptprg{-{}#2}%
      }%
      {%
        \mbox{\textcolor{gray}{\texttt{-{}#2}}}%
        \index{#2@\texttt{-{}#2}}\index{Program options!#1@\texttt{-{}#2}}%
      }%
  }%

% \DescribeIF{m}, #1 image format
\newsavebox{\marginIF}
\NewDocumentCommand{\DescribeIF}{ m }
  {%
    \begin{lrbox}{\marginIF}%
      \begin{minipage}[t]{\marginparwidth}%
        \raggedleft\ttfamily\bfseries%
        \textcolor{optcolor}{#1}\hfill
      \end{minipage}%
    \end{lrbox}%
    \index{Image format!#1@\texttt{#1}}%
    \leavevmode%
    \marginpar{\usebox{\marginIF}}%
    \ignorespaces%
  }%

% \myenv{environ} for margin
\NewDocumentCommand{\myenv}{m}
  {
    \begin{minipage}[t]{\marginparwidth}%
        \raggedleft\ttfamily\small%
        {\textcolor{gray}{\textbackslash begin\{}}{\bfseries\textcolor{pkgcolor}{#1}}\textcolor{gray}{\}}\par%
        \meta[ac=lightgray,cf=gray]{env content}\par%
        {\textcolor{gray}{\textbackslash end\{}}{\bfseries\textcolor{pkgcolor}{#1}}\textcolor{gray}{\}}%
    \end{minipage}%
    \index{Environment suport by default!#1@\texttt{#1}}
  }
% CleanPST for margin
\NewDocumentCommand{\mytag}{m}
  {%
   \begin{minipage}[t]{\marginparwidth}%
        \raggedleft\ttfamily\small%
        \textcolor{gray}{\%}{\bfseries\textcolor{pkgcolor}{#1}}\par%
        \meta[ac=lightgray,cf=gray]{code}\par%
        \textcolor{gray}{\%}{\bfseries\textcolor{pkgcolor}{#1}}%
    \end{minipage}%
    \index{Remove PST code!#1@\texttt{#1}}
  }%

% \DescribeTE{sm}, #1 tag, #2 env
\newsavebox{\marginenvtag}
\NewDocumentCommand{\DescribeTE}{sm}{%
\begin{lrbox}{\marginenvtag}%
   \begin{minipage}[t]{\marginparwidth}%
    \raggedleft
    \IfBooleanTF{#1}{\mytag{#2}}{\myenv{#2}}
    \end{minipage}%
\end{lrbox}%
    \leavevmode%
    \marginpar{\usebox{\marginenvtag}}%
    \ignorespaces%
}%
\makeatother

\ExplSyntaxOn
% \cmdopt[short]{long}
\NewDocumentCommand{\cmdopt}{om}
  {
    \IfNoValueTF{#1}
      {
        \textcolor{optcolor}{\texttt{-\/-#2}}
      }
      {
        \textcolor{optcolor}{\texttt{-{}#1}},
        \textcolor{optcolor}{\texttt{-\/-#2}}
      }
    \Loptcmd{-\/-#2}
  }

% Custom \meta[...]{...}, \marg[...]{...} and \oarg[...]{...} for color
\NewDocumentCommand{\meta}{O{}m}
  {
   \myscript_meta_generic:Nnn \myscript_meta:n { #1 } { #2 }
  }
\NewDocumentCommand{\marg}{O{}m}
  {
   \myscript_meta_generic:Nnn \myscript_marg:n { #1 } { #2 }
  }
\NewDocumentCommand{\oarg}{O{}m}
  {
   \myscript_meta_generic:Nnn \myscript_oarg:n { #1 } { #2 }
  }
% Variables and keys
\tl_new:N \l_myscript_meta_font_tl

\keys_define:nn { myscript/meta }
  {
   type .choice:,
   type / tt .code:n = \tl_set:Nn \l_myscript_meta_font_tl { \ttfamily },
   type / rm .code:n = \tl_set:Nn \l_myscript_meta_font_tl { \rmfamily },
   type .initial:n = rm,
   cf .tl_set:N = \l_myscript_meta_color_tl,
   cf .initial:n = black,%
   ac .tl_set:N = \l_myscript_meta_anglecolor_tl,
   ac .initial:n = lightgray,
   sbc .tl_set:N = \l_myscript_meta_brackcolor_tl,
   sbc .initial:n = gray,
   cbc .tl_set:N = \l_myscript_meta_bracecolor_tl,
   cbc .initial:n = gray,
  }
% Internal commands
\cs_new_protected:Npn \myscript_meta_generic:Nnn #1 #2 #3
  {
   \group_begin:
    \keys_set:nn { myscript/meta } { #2 }
    \color{ \l_myscript_meta_color_tl }
    \l_myscript_meta_font_tl
    #1 { #3 } % #1 is \myscript_meta:n, \myscript_marg:n or \myscript_oarg:n
   \group_end:
  }
\cs_new_protected:Npn \myscript_meta:n #1
  {
   \myscript_meta_angle:n { \textlangle }
   \myscript_meta_meta:n { #1 }
   \myscript_meta_angle:n { \textrangle }
  }
\cs_new_protected:Npn \myscript_marg:n #1
  {
   \myscript_meta_brace:n { \textbraceleft }
   \myscript_meta:n { #1 }
   \myscript_meta_brace:n { \textbraceright }
  }
\cs_new_protected:Npn \myscript_oarg:n #1
  {
   \myscript_meta_brack:n { [ }
   \myscript_meta:n { #1 }
   \myscript_meta_brack:n { ] }
  }
\cs_new_protected:Npn \myscript_meta_meta:n #1
  {
   \textnormal{\textit{#1}}
  }
\cs_new_protected:Npn \myscript_meta_angle:n #1
  {
   \group_begin:
    \fontfamily{lmr}\selectfont
    \textcolor{\l_myscript_meta_anglecolor_tl}{#1}
   \group_end:
  }
\cs_new_protected:Npn \myscript_meta_brace:n #1
  {
   \group_begin:
    \color{\l_myscript_meta_bracecolor_tl}
    #1
   \group_end:
  }
\cs_new_protected:Npn \myscript_meta_brack:n #1
  {
   \textcolor{\l_myscript_meta_brackcolor_tl}{#1}
  }

% \DescribeCmd[...]{...}{...}{...}
\newsavebox{\optcmdline}
\NewDocumentCommand{\DescribeCmd}{ o m m m }
  {
  \group_begin:
    \begin{lrbox}{\optcmdline}%
      \begin{minipage}[t]{\marginparwidth}%
      \ttfamily\raggedleft%
      \IfNoValueTF{#1}
        {\textcolor{optcolor}{-\/-#2}}
        {\textcolor{optcolor}{-{#1}}\textcolor{gray}{,} \textcolor{optcolor}{-\/-#2}}%
      \end{minipage}%
      \Loptcmd{-\/-#2}
    \end{lrbox}%
    \leavevmode%
    \marginpar{\usebox{\optcmdline}}%
    \meta[ac=lightgray,cf=gray]{#3}
    \raggedleft\hfill\textcolor{gray}{\textsf{(default: ~ {#4})}}\ignorespaces\par%
    %\vspace*{2pt}\par%
  \group_end:
}

\ExplSyntaxOff
% Don't copy numbers in code example for listings
\newcommand*{\noaccsupp}[1]{\BeginAccSupp{ActualText={}}#1\EndAccSupp{}}

% Create a language for documentation
\lstdefinelanguage{myscript-doc}{
    texcsstyle=*,%
    escapechar=`,%
    showstringspaces=false,%
    extendedchars=true, %
    stringstyle = {\color{red}},%
    alsoletter={\-,.},%
% comments
    morecomment=[l]{\%},%
    commentstyle=\lmmitalic\color{lightgray},%
% Important words 1
    keywordstyle=[1]{\color{NavyBlue}},%
    keywords=[1]{PassOptionsToPackage,AtBeginDocument,documentclass,usepackage,section},%
% Important words 2 (macros)
    keywordstyle=[2]{\color{blue!75}},%
    keywords=[2]{graphicspath,RequirePackage,renewcommand,includegraphics,coordinate,draw,approx,%
    PreviewBbAdjust,setlength,parindent,usetikzlibrary,tikzexternalize,psset,tikzset,PrependGraphicsExtensions,%
    DefineShortVerb,lstMakeShortInline,MakeSpecialShortVerb,UndefineShortVerb,verb,myverb,macro,put,line},%
% Important words 3, options in input file
    keywordstyle=[3]{\color{optcolor!85}},%
    keywords=[3]{arara,extensions,luatex,env,norun,%
    clean,pst,tkz,eps,pdf,xetex,latex,luatex,dvips,png,latexmk,svg,srcenv,noprew,imgdir},%
% Important words 4, environments and tags
    keywordstyle=[4]{\color{pkgcolor}},%
    keywords=[4]{pspicture,endpspicture,description,filecontents,PSTexample,%
    psgraph, endpsgraph,CleanPST,postscript},%
% Important words 5, directory
    keywordstyle=[5]{\color{OrangeRed}},%
    keywords=[5]{images},% ,pics
% Important words 6, begin, end and other
    keywordstyle=[6]{\color{gray}},% replace this color
    keywords=[6]{begin,end,-recorder,doc,active,inactive,tightpage,BEGIN,END,%
    -shell-escape,osf,--output-format,dvi},%
% Important words 7, reserved for arara
    keywordstyle=[7]{\color{araracolor}},%
    keywords=[7]{lualatex,halt,xelatex},%
% Important words 8, pkg and others load in examples
    keywordstyle=[8]{\color{SlateBlue}},%
    keywords=[8]{graphicx,grfext,article,document,preview,libertinus,pst-pdf},%
% Important words 9, files used in examples
    keywordstyle=[10]{\color{OrangeRed}},%
    keywords=[10]{test-fig-all.pdf,test-fig-1981.tex,test-fig-1981.pdf, test-fig-1,test-fig-2,test-out,test-in.ltx,%
    test-fig-1981.tex,test-out.tex,file-in.tex,file-out,test.tex,test-fig-all.dvi,test-fig-1.svg,test-fig-all.tex},%
}[keywords,tex,comments,strings]% end languaje

% \begin{examplecode}[opts]...\end{examplecode}
\lstnewenvironment{examplecode}[1][]{%
\lstset{
    language=myscript-doc,%
    stringstyle = {\color{red}},%
    basicstyle=\ttfamily\small,%
    numbersep=1em,%
    numberstyle=\tiny\color{gray}\noaccsupp,%
    rulecolor=\color{gray!50},%
    framesep=\fboxsep,%
    framerule=\fboxrule,%
    xleftmargin=\dimexpr\fboxsep+\fboxrule\relax,%
    xrightmargin=\dimexpr\fboxsep+\fboxrule\relax,%
% literateee
    literate=*{\{}{{\bfseries\textcolor{gray}{\{}}}{1}
              {\}}{{\bfseries\textcolor{gray}{\}}}}{1}
              {[}{{\textcolor{gray}{[}}}{1}
              {]}{{\textcolor{gray}{]}}}{1}
              {\*}{{\bfseries\textcolor{MediumOrchid}{*}}}{1}
              {:}{{\textcolor{gray}{:}}}{1}
              {,}{{\textcolor{gray}{,}}}{1}
              {=}{{\textcolor{gray}{=}}}{1}
              {/}{{\textcolor{OrangeRed}{/}}}{1}
              {scale=1}{{\textcolor{gray}{scale=1}}}{7}
              {\%CleanPST}{{\textcolor{gray}{\%}\textcolor{pkgcolor}{CleanPST}}}{9},%
          #1,%
    }% close lstset
}%
{}% close examplecode

% \begin{examplecmd}...\end{examplecmd}
\lstnewenvironment{examplecmd}{%
\lstset{
    language=myscript-doc,%
    basicstyle=\ttfamily\small,%
    frame=single,%
    alsoletter={\-,.,\-\-},%
    rulecolor=\color{gray!50},%
    framesep=\fboxsep,%
    framerule=\fboxrule,%
    xleftmargin=\dimexpr\fboxsep+\fboxrule\relax,%
    xrightmargin=\dimexpr\fboxsep+\fboxrule\relax,%
    keywordstyle=[20]{\color{gray}},% replace this color
    keywords=[20]{--margins, -recorder,-shell-escape,-s,-o,-e,-p,-j,--imgdir,%
    --luatex,--latexmk,--svg,--png,--nopdf,pics,-no-shell-escape,-interaction,nonstopmode},%
% Reserved words (cmd line options)
    deletekeywords=[4]{pst2pdf},%
    classoffset=7,%
    keywordstyle=\bfseries\color{ForestGreen},%
    morekeywords={pst2pdf},%
% % Reserved words (cmd line options)
    classoffset=8,%
    keywordstyle={\color{ForestGreen}},%
    morekeywords={gs,pdftoppm,pdftocairo,pdftops,cd,perl,pdfcrop},%
% Only for command line options
    classoffset=5,%
    keywordstyle=\color{blue},%
    keywords={user,machine},%
    literate=*{[}{{\textcolor{darkgray}{[}}}{1}
              {]}{{\textcolor{darkgray}{]}}}{1}
              {=}{{\textcolor{gray}{=}}}{1}
              {@}{{\textcolor{blue}{@}}}{1}
              {\$}{{\textcolor{blue}{\$}}}{1}
              {:}{{\textcolor{blue}{:}}}{1}
              {§}{{\textcolor{red}{\$\space}}}{2}
              {~}{{\textcolor{blue}{\bfseries\textasciitilde}}}{1}%
    }% close lstset
}%
{}% close examplecmd

% \lstinline[style=inline]|...|
\lstdefinestyle{inline}
  {
   language=myscript-doc,%
   basicstyle=\ttfamily\color{gray},%
   escapechar=`,%
   upquote=true,%
   deletekeywords=[8]{preview},%
   morekeywords =[4]{preview},%
   literate=*{\%}{{\bfseries\textcolor{gray}{\%}}}{1}
             {\*}{{\bfseries\textcolor{MediumOrchid}{*}}}{1}
  }

% set default style
\lstset{style=inline}

\begin{document}

\title{\texttt{pst2pdf}}
\subtitle{Running a PSTricks document with pdflatex;\\  \small v\fileversion}
\author{Herbert Voß  \\ Pablo Gonz\'{a}lez Luengo}
\docauthor{}
\date{\filedate}

\maketitle

\begin{abstract}
    \texttt{pst2pdf} run a \TeX{} source, read all \texttt{postscript}, \texttt{pspicture}, \texttt{psgraph}
    and \texttt{PSTexample} environments and convert in images \texttt{pdf,eps,jpg,svg}
    or \texttt{png} format (default \texttt{pdf}), extract source code in standalone files and create new
    file with all environments converted to \Lcs{includegraphics} and runs \texttt{(pdf/Xe/lua)latex}.
\end{abstract}

\tableofcontents

\clearpage
\newpage

\section{Introduction}

\PST\ as \PS\ related package uses the programming language \PS\ for internal
calculations. This is an important advantage, because floating point arithmetic is no
problem. Nearly all mathematical calculation can be done when running the \DVI-file
with \gs. However, creating a \PDF\ file in a direct way with \Lprog{pdflatex} is
not possible. \Lprog{pdflatex} cannot understand the \PS\ related stuff.

Instead of running \Lprog{pdflatex} one can use the \emph{script} \Lscript{pst2pdf}, it extracts
all \PST\ related code into single documents with the same preamble as the original
main document.

The \Lscript{pst2pdf} \emph{script} runs document, clips all whitespace around the
image and creates a \Lext{pdf} images of the \PST\ related code.

In a last run which is the \Lprog{pdflatex} the \PST\ code in the
main document is replaced by the created images.

\section{License}

This program is free software; you can redistribute it and/or modify
it under the terms of the \href{https://www.gnu.org/licenses/gpl-3.0.html}{GNU General Public License} as published by
the \href{https://www.fsf.org/}{Free Software Foundation}; either version 3 of the License, or
(at your option) any later version.


This program is distributed in the hope that it will be useful, but
WITHOUT ANY WARRANTY; without even the implied warranty of
MERCHANTABILITY or FITNESS FOR A PARTICULAR PURPOSE. See the \href{https://www.gnu.org/licenses/gpl-3.0.html}{GNU General Public License}
for more details.

\section{Requirements for operation}

For the complete operation of \Lscript{pst2pdf}{} you need to have a modern %
\hologo{TeX} distribution such as \hologo{TeX} Live or \hologo{MiKTeX}, the packages \LPack{preview}\cite{preview},
\LPack{pst-pdf}\cite{pst-pdf}, \LPack{graphicx}\cite{graphicx} and \LPack{grfext}\cite{grfext}, have
a version equal to or greater than \liningnums{5.28} of \Lprog{perl}, a
version equal to or greater than \liningnums{9.24} of \Lprog{Ghostscript}, a
version equal to or greater than \liningnums{1.40} of \Lscript{pdfcrop}
and have a version equal to or greater than \liningnums{0.52} of %
\Lprog{poppler-utils}.

The distribution of \hologo{TeX} Live 2020 for \Losystem{Windows} includes %
\Lscript{pst2pdf} and all requirements, \hologo{MiKTeX} users must install the
appropriate software for full operation.

The script auto detects the \Lprog{Ghostscript}, but not \Lprog{poppler-utils}.
You should keep this in mind if you are using the script directly and not
the version provided in your \hologo{TeX} distribution.

The script has been tested on \Losystem{Windows} (v10) and %
\Losystem{Linux} (fedora 32) using \Lprog{Ghostscript} %
\liningnums{v9.52}, \Lprog{poppler-utils} \liningnums{v0.84}, \Lprog{perl} \liningnums{v5.30}
and the standard classes offers by \hologo{LaTeX}: \LClass{book},
\LClass{report}, \LClass{article} and \LClass{letter}.

\section{How it works}
\label{sec:howtowork}

It is important to have a general idea of how the \emph{\enquote{extraction and conversion}}
process works and the requirements that must be fulfilled so
that everything works correctly, for this we must be clear about some
concepts related to how to work with the \meta{input file}, the \meta{verbatim content}
and the \meta{steps process}.

\subsection{The input file}
\label{sec:inputfile}

The \meta{input file} must comply with \emph{certain characteristics} in order to
be processed, the content at the beginning and at the end of the \meta{input
file} is treated in a special way, before \lstinline|\documentclass| and after
\lstinline|\end{document}| can go any type of content, internally the script will
\emph{\enquote{split}} the \meta{input file} at this points.

If the \meta{input file} contains files using \Lcs{input}\marg[type=tt]{file} or
\Lcs{include}\marg[type=tt]{file} these will not be processed, from the side of
the \emph{script} they only represent lines within the file, if you want
them to be processed it is better to use the \Lscript{latexpand}\footnote{\url{https://www.ctan.org/pkg/latexpand}}
first and then process the file.

Like \Lcs{input}\marg[type=tt]{file} or \Lcs{include}\marg[type=tt]{file}, blank lines, vertical spaces and tab
characters are treated literally, for the \emph{script} the \meta{input file}
is just a set of characters, as if it was a simple text file. It is
advisable to format the source code \meta{input file} using utilities such
as \Lprog{chktex}\footnote{\url{https://www.ctan.org/pkg/chktex}} and
\Lscript{latexindent}\footnote{\url{https://www.ctan.org/pkg/latexindent}}, especially if you want to
extract the source code of the environments.

Both \Lcs{thispagestyle}\marg[type=tt]{style} and \Lcs{pagestyle}\marg[type=tt]{style}
are treated in a special way by the script, if they do not appear in the
preamble then \Lcs{pagestyle}\marg[type=tt]{empty} will be added and if they are present
and \marg[type=tt]{style} is different from \marg[type=tt]{empty} this
will be replaced by \marg[type=tt]{empty}.

This is necessary for the image creation process, it does not affect the \meta{output file}, but
it does affect the \emph{standalone} files. For the script the process of dividing the \meta{input file} into four
parts and then processing them:

\begin{examplecode}[numbers=left,frame=single]
% Part One: Everything before \documentclass
\documentclass{article}
% Part two: Everything between \documentclass and \begin{document}
\begin{document}
% Part three: : Everything between \begin{document} and \end{document}
\end{document}
% Part Four: Everything after \end{document}
\end{examplecode}

If for some reason you have an environment \Lenv{filecontens} before \lstinline|\documentclass| or
in the preamble of the \meta{input file} that contains a \emph{sub-document} or \emph{environment}
you want to extract, the script will ignore them.

\subsection{Verbatim contents}
\label{sec:verbatim}

One of the greatest capabilities of this script is to \emph{\enquote{skip}}
the complications that \meta{verbatim content} produces with the extraction
of environments using tools outside the \enquote{\hologo{TeX} world}.
In order to \emph{\enquote{skip}} the complications, the \meta{verbatim content} is classified into
three types:

\begin{itemize}[nosep]
    \item Verbatim in line.
    \item Verbatim standard.
    \item Verbatim write.
\end{itemize}

\subsubsection{Verbatim in line}
\label{sec:verbatim:inline}

The small pieces of code written using a \emph{\enquote{verbatim macro}}
are considered \meta{verbatim in line}, such as \lstinline+\verb|+\meta{code}\lstinline+|+
or \lstinline+\verb*|+\meta{code}\lstinline+|+ or \lstinline+\macro+\marg[type=tt]{code} or
\lstinline+\macro+\oarg[type=tt]{opts}\marg[type=tt]{code}.

Most \emph{\enquote{verbatim macro}} provide by packages \LPack{minted}, %
\LPack{fancyvrb} and \LPack{listings} have been tested and are fully
supported. They are automatically detected the \emph{verbatim macro} (including \texttt{\small\bfseries\textcolor{MediumOrchid}{*}} argument) generates by
\Lcs{newmint} and \Lcs{newmintinline} and the following list:

\vspace*{-10pt}
\begin{multicols}{3}
\begin{itemize}[font=\sffamily\small,partopsep=5pt,parsep=5pt,nosep,leftmargin=*]
\small
\item \Lcs{mint}
\item \Lcs{spverb}
\item \Lcs{qverb}
\item \Lcs{fverb}
\item \Lcs{verb}
\item \Lcs{Verb}
\item \Lcs{lstinline}
\item \Lcs{pyginline}
\item \Lcs{pygment}
\item \Lcs{Scontents}
\item \Lcs{tcboxverb}
\item \Lcs{mintinline}
\end{itemize}
\end{multicols}

Some packages define abbreviated versions for \emph{\enquote{verbatim macro}} as %
\Lcs{DefineShortVerb}, \Lcs{lstMakeShortInline} and %
\Lcs{MakeSpecialShortVerb}, will be detected automatically if are declared
explicitly in \meta{input file}.

The following consideration should be kept in mind for some packages that
use abbreviations for verbatim macros, such as \LPack{shortvrb} or %
\LPack{doc} for example in which there is no explicit \lstinline+\macro+ in the
document by means of which the abbreviated form can be detected, for
automatic detection need to find \Lcs{DefineShortVerb} explicitly to process
it correctly. The solution is quite simple, just add in \meta{input file}:

\begin{examplecode}
\UndefineShortVerb{\|}
\DefineShortVerb{\|}
\end{examplecode}

depending on the package you are using. If your \emph{\enquote{verbatim macro}} is not
supported by default or can not detect, use the options described in \ref%
{sec:optline}.

\subsubsection{Verbatim standard}
\label{sec:verbatim:std}

These are the \emph{\enquote{classic}} environments for \emph{\enquote{writing code}} are considered %
\meta{verbatim standard}, such as \Venv{verbatim} and \Venv{lstlisting}
environments. The following list (including \texttt{\small\bfseries\textcolor{MediumOrchid}{*}} argument)
is considered as \meta{verbatim standard} environments:

\begin{multicols}{4}
\begin{itemize}[font=\sffamily\small, noitemsep,leftmargin=*]
\small
\item \Venv{Example}
\item \Venv{CenterExample}
\item \Venv{SideBySideExample}
\item \Venv{PCenterExample}
\item \Venv{PSideBySideExample}
\item \Venv{verbatim}
\item \Venv{Verbatim}
\item \Venv{BVerbatim}
\item \Venv{LVerbatim}
\item \Venv{SaveVerbatim}
\item \Venv{PSTcode}
\item \Venv{LTXexample}
\item \Venv{tcblisting}
\item \Venv{spverbatim}
\item \Venv{minted}
\item \Venv{listing}
\item \Venv{lstlisting}
\item \Venv{alltt}
\item \Venv{comment}
\item \Venv{chklisting}
\item \Venv{verbatimtab}
\item \Venv{listingcont}
\item \Venv{boxedverbatim}
\item \Venv{demo}
\item \Venv{sourcecode}
\item \Venv{xcomment}
\item \Venv{pygmented}
\item \Venv{pyglist}
\item \Venv{program}
\item \Venv{programl}
\item \Venv{programL}
\item \Venv{programs}
\item \Venv{programf}
\item \Venv{programsc}
\item \Venv{programt}
\end{itemize}
\end{multicols}

They are automatically detected \meta{verbatim standard} environments
(including \texttt{\small\bfseries\textcolor{MediumOrchid}{*}} argument)
generates by commands:

\begin{multicols}{2}
\begin{itemize}[font=\sffamily\small,noitemsep,leftmargin=*]
\small
\item \Lcs{DefineVerbatimEnvironment}
\item \Lcs{NewListingEnvironment}
\item \Lcs{DeclareTCBListing}
\item \Lcs{ProvideTCBListing}
\item \Lcs{lstnewenvironment}
\item \Lcs{newtabverbatim}
\item \Lcs{specialcomment}
\item \Lcs{includecomment}
\item \Lcs{newtcblisting}
\item \Lcs{NewTCBListing}
\item \Lcs{newverbatim}
\item \Lcs{NewProgram}
\item \Lcs{newminted}
\end{itemize}
\end{multicols}

If any of the \meta{verbatim standard} environments is not supported by
default or can not detected, you can use the options described in \ref%
{sec:optline}.

\subsubsection{Verbatim write}
\label{sec:verbatim:write}

Some environments have the ability to write \emph{\enquote{external files}}
or \emph{\enquote{store content}} in memory, these environments are
considered \meta{verbatim write}, such as \Wenv{scontents}, \Wenv{filecontents}
or \Lenv{VerbatimOut} environments. The following list is considered (including \texttt{\small\bfseries\textcolor{MediumOrchid}{*}} argument) as %
\meta{verbatim write} environments:

\begin{multicols}{4}
\begin{itemize}[font=\sffamily\small, noitemsep,leftmargin=*]
\ttfamily\small
\item \Wenv{scontents}
\item \Wenv{filecontents}
\item \Wenv{tcboutputlisting}
\item \Wenv{tcbexternal}
\item \Wenv{tcbwritetmp}
\item \Wenv{extcolorbox}
\item \Wenv{extikzpicture}
\item \Wenv{VerbatimOut}
\item \Wenv{verbatimwrite}
\item \Wenv{filecontentsdef}
\item \Wenv{filecontentshere}
\item \Wenv{filecontentsdefmacro}
\item \Wenv{filecontentsdefstarred}
\item \Wenv{filecontentsgdef}
\item \Wenv{filecontentsdefmacro}
\item \Wenv{filecontentsgdefmacro}
\end{itemize}
\end{multicols}

They are automatically detected \meta{verbatim write}
(including \texttt{\small\bfseries\textcolor{MediumOrchid}{*}} argument) environments generates
by commands:

\begin{itemize}[font=\sffamily\small, noitemsep,leftmargin=*]
\small
\item \Lcs{renewtcbexternalizetcolorbox}
\item \Lcs{renewtcbexternalizeenvironment}
\item \Lcs{newtcbexternalizeenvironment}
\item \Lcs{newtcbexternalizetcolorbox}
\item \Lcs{newenvsc}
\end{itemize}

If any of the \meta{verbatim write} environments is not supported by default
or can not detected, you can use the options described in \ref{sec:optline}.

\subsection{Steps process}
\label{sec:steps:process}

For creation of the image formats, extraction of source code of environments and creation of an
\meta{output file}, \Lscript{pst2pdf} need a various steps. Let's assume that the %
\meta{input file} is \LFile{test.tex}, \meta{output file} is %
\LFile{test-pdf.tex}, the working directory are \enquote{\texttt{./}}, the
directory for images are \texttt{./images}, the temporary
directory is \texttt{/tmp} and we want to generate images in \Limgform{pdf}
format and \meta{standalone} files for all environments extracted.

We will use the following code as \LFile{test.tex}:

\begin{examplecode}[numbers=left,frame=single]
% Some commented lines at begin file
\documentclass{article}
\usepackage{pstricks}
\begin{document}
Some text
\begin{pspicture}
  Some code
\end{pspicture}
Always use \verb|\begin{pspicture}| and \verb|\end{pspicture}| to open
and close environment
\begin{pspicture}
  Some code
\end{pspicture}
Some text
\begin{verbatim}
\begin{pspicture}
  Some code
\end{pspicture}
\end{verbatim}
Some text
\end{document}
Some lines that will be ignored by the script
\end{examplecode}

\subsubsection{Validating Options}

The first step is read and validated \oarg[type=tt,cf=optcolor,sbc=gray,ac=lightgray]{options} from the command
line, verifying that \LFile{test.tex} contains \emph{some} environment to extract,
check the directory \texttt{./images} if it doesn't exist create it and create a temporary directory \texttt{/tmp/hG45uVklv9}.

The entire \LFile{test.tex} file is loaded into memory and \emph{\enquote{split}} to start the extraction process.

\subsubsection{Comment and ignore}
In the second step, once the file \LFile{test.tex} is loaded and divided in memory,
proceeds (in general terms) as follows:

\begin{quotation}
Search the words \lstinline|\begin{| and \lstinline|\end{| in verbatim standard, verbatim write, verbatim in line and
commented lines, if it finds them, converts to \lstinline|\BEGIN{| and \lstinline|\END{|, then places all code to
extract inside the \lstinline|\begin{preview}| \ldots \lstinline|\end{preview}|.
\end{quotation}

At this point \emph{\enquote{all}} the code you want to extract is inside \lstinline|\begin{preview}|\ldots\lstinline|\end{preview}|.

\subsubsection{Creating standalone files and extracting}

In the third step, the script generate \meta{standalone} files: \LFile{test-fig-1.tex},
\LFile{test-fig-2.tex}, \dots{} and saved in \texttt{./images} then proceed in two ways
according to the \oarg[type=tt,cf=optcolor,sbc=gray,ac=lightgray]{options}
passed to generate a temporary file with a random number (1981 for example):

\begin{enumerate}[leftmargin=*]
\item If script is call \emph{without} \cmdopt{noprew} options, the following lines
will be added at the beginning of the \LFile{test.tex} (in memory):

\begin{examplecode}
\PassOptionsToPackage{inactive}{pst-pdf}%
\AtBeginDocument{%
\RequirePackage[inactive]{pst-pdf}%
\RequirePackage[active,tightpage]{preview}%
\renewcommand\PreviewBbAdjust{-60pt -60pt 60pt 60pt}}%
% rest of input file
\end{examplecode}

The different parts of the file read in memory are joined  and save in a temporary
file \LFile{test-fig-1981.tex} in \texttt{./}. This file will contain all the environments
for extraction between \lstinline|\begin{preview}|\ldots\lstinline|\end{preview}| along with the rest of the document.
If the document contains images, these must be in the formats supported by the \emph{engine} selected to process the \meta{input file}.

\item If script is call \emph{with} \cmdopt{noprew} options, the \lstinline|\begin{preview}|\ldots\lstinline|\end{preview}|
lines are only used as delimiters for extracting the content \emph{without} using the package \LPack{preview}, the following lines
will be added at the beginning of the \LFile{test.tex} (in memory):

\begin{examplecode}
\PassOptionsToPackage{inactive}{pst-pdf}%
\AtBeginDocument{%
\RequirePackage[inactive]{pst-pdf}}%
% only environments extracted
\end{examplecode}

Then it is joined with all extracted environments separated by \Lcs{newpage} and saved
in a temporary file \LFile{test-fig-1981.tex} in \enquote{\texttt{./}}.

\end{enumerate}

If \cmdopt{norun} is passed, the temporary file \LFile{test-fig-1981.tex}
is renamed to \LFile{test-fig-all.tex} and moved to \texttt{./images}.

\subsubsection{Generate image formats}

In the fourth step, the script generating the file \LFile{test-fig-1981.pdf} with all code extracted and croping, running:
\begin{examplecmd}
[user@machine ~:]§ `\small\meta[type=tt,cf=ForestGreen,ac=lightgray]{compiler}` -no-shell-escape -interaction=nonstopmode -recorder test-fig-1981.tex
[user@machine ~:]§ pdfcrop --margins `\small\textcolor{gray}{0}` test-fig-1981.pdf test-fig-1981.pdf
\end{examplecmd}

Now move \LFile{test-fig-1981.pdf} to \texttt{/tmp/hG45uVklv9} and rename to \LFile{test-fig-all.pdf}, generate image files \LFile{test-fig-1.pdf}
and \LFile{test-fig-2.pdf} and copy to \texttt{./images}. The file \LFile{test-fig-1981.tex} is
moved to the \texttt{./images} and rename to \LFile{test-fig-all.tex}.

Note the options passed to \meta[type=tt,cf=ForestGreen,ac=lightgray]{compiler} always use \prgopt*{no-shell-escape}
and \prgopt*{recorder}, to generate the \Lext{fls} file which is used to delete temporary files and directories after
the process is completed and. The \cmdopt{shell} option activates \prgopt*{shell-escape} for compatibility with packages
such as \LPack{minted} or others.

\subsubsection{Create output file}

In the fifth step, the script creates the output file \LFile{test-pdf.tex} converting all extracted code to
\Lcs{includegraphics}, remove all \PST{} packages and content betwen \lstinline|%CleanPST ... %CleanPST|, then
adding the following lines at end of preamble:

\begin{examplecode}[numbers=left]
\usepackage{graphicx}
\graphicspath{{images/}}
\usepackage{grfext}
\PrependGraphicsExtensions*{.pdf}
\end{examplecode}

The script will try to detect whether the \LPack{graphicx} package and
the \Lcs{graphicspath} command are in the preamble of the \meta[cf=optcolor]{output file}.
If it is not possible to find it, it will read the \Lext{log} file generated by the temporary file.
Once the detection is complete, the package \LPack{grfext} and \Lcs{PrependGraphicsExtensions*}
will be added at the end of the preamble, then proceed to run:

\begin{examplecmd}
[user@machine ~:]§ `\small\meta[type=tt,cf=ForestGreen,ac=lightgray]{compiler}` -recorder -shell-escape test-pdf.tex
\end{examplecmd}
generating the file \LFile{test-pdf.pdf}.

\subsubsection{Clean temporary files and dirs}

In the sixth step, the script read the files \LFile{test-fig-1981.fls}
and \LFile{test-out.fls}, extract the information from the
temporary files and dirs generated in the process in \enquote{\texttt{./}} and then
delete them together with the directory \texttt{/tmp/hG45uVklv9}.

Finally the output file \texttt{test-pdf.tex} looks like this:

\begin{examplecode}[numbers=left]
% some commented lines at begin document
\documentclass{article}
\usepackage{graphicx}
\graphicspath{{images/}}
\usepackage{grfext}
\PrependGraphicsExtensions*{.pdf}
\begin{document}
Some text
\includegraphics[scale=1]{test-fig-1}
Always use \verb|\begin{pspicture}| and \verb|\end{pspicture}| to open
and close environment
\includegraphics[scale=1]{test-fig-2}
Some text
\begin{verbatim}
\begin{pspicture}
  Some code
\end{pspicture}
\end{verbatim}
Some text
\end{document}
\end{examplecode}

\section{Default extracted environments}
\label{sec:extract:env}

\nxLprog{pst2pdf} support fourth environments for extraction. Internally
the script converts all environments to extract in \Lenv{preview} environments.
Is better comment this package in preamble unless the option \cmdopt{noprew} is used.

\DescribeTE{postscript}
Environment provide by \LPack{pst-pdf}\cite{pst-pdf}, \LPack{auto-pst-pdf}\cite{auto-pst-pdf}
and \LPack{auto-pst-pdf-lua}\cite{auto-pst-pdf-lua} packages. Since the \LPack{pst-pdf}, \LPack{auto-pst-pdf}
and \LPack{auto-pst-pdf-lua} packages internally use the \LPack{preview} package, is better comment this in preamble.
Only the \emph{content} of this environment is extracted and \emph{\enquote{not}} the environment itself when using the
\cmdopt{srcenv} option. The \Lenv{postscript} environment should always be used, when there is some code before a \Lenv{pspicture}
environment or for some code which is not inside of a \Lenv{pspicture} environment.

\DescribeTE{pspicture}
Environment provide by \LPack{pstricks}\cite{girou:02:} package. The plain \hologo{TeX}
syntax \lstinline|\pspicture ... \endpspicture| its converted to \hologo{LaTeX} syntax
\lstinline|\begin{pspicture} ... \end{pspicture}| if not within the \Lenv{PSTexample} or \Lenv{postscript} environments.

\DescribeTE{psgraph}
Environment provide by \LPack{pst-plot}\cite{pst-plot} package. The plain \hologo{TeX} syntax \lstinline|\psgraph ... \endpsgraph|
its converted to \hologo{LaTeX} syntax \lstinline|\begin{psgraph} ... \end{psgraph}| if
not within the \Lenv{PSTexample} or \Lenv{postscript} environments.

\DescribeTE{PSTexample}
Environment provide by \LPack{pst-exa}\cite{pst-exa} package. The script automatically detects the
\lstinline|\begin{PSTexample}|  \lstinline|...\end{PSTexample}|
environments and processes them as separately compiled files. The user should have loaded the
package with [\Loption{swpl}] or [\Loption{tcb}] option.

\section{Remove \PST\ code}

By design, the script remove all \PST{} packages in preamble of \meta{output file}, if you need delete other \PST\ code in preamble use:

\DescribeTE*{CleanPST}
All content betwen \lstinline|%CleanPST ... %CleanPST| are deleted in preamble of the \meta[cf=optcolor]{output file}.
This lines can \emph{not} be nested and should be at the beginning of the line and in separate lines.

\section{Supported image formats}
\label{sec:image:format}

The \Larga{image formats} generated by the \Lscript{pst2pdf} using \Lprog{Ghostscript}
and \Lprog{poppler-utils} are the following command lines:

\DescribeIF{pdf}
The image format generated using \Lprog{Ghostscript}. The line executed by the system is:

\begin{examplecmd}
[user@machine ~:]§ gs -q -dNOSAFER -sDEVICE=pdfwrite -dPDFSETTINGS=/prepress
\end{examplecmd}

\DescribeIF{eps}
The image format generated using \Lprog{pdftoeps}. The line executed by the system is:

\begin{examplecmd}
[user@machine ~:]§ pdftops -q -eps
\end{examplecmd}

\DescribeIF{png}
The image format generated using \Lprog{Ghostscript}. The line executed by the system is:

\begin{examplecmd}
[user@machine ~:]§ gs -q -dNOSAFER -sDEVICE=pngalpha -r150
\end{examplecmd}

\DescribeIF{jpg}
The image format generated using \Lprog{Ghostscript}. The line executed by the system is:
\begin{examplecmd}
[user@machine ~:]§ gs -q -dNOSAFER -sDEVICE=jpeg -r150 -dJPEGQ=100 \
                      -dGraphicsAlphaBits=4 -dTextAlphaBits=4
\end{examplecmd}

\DescribeIF{ppm}
The image format generated using \Lprog{pdftoppm}. The line executed by the system is:

\begin{examplecmd}
[user@machine ~:]§ pdftoppm -q -r 150
\end{examplecmd}

\DescribeIF{tiff}
The image format generated using \Lprog{Ghostscript}. The line executed by the system is:

\begin{examplecmd}
[user@machine ~:]§ gs -q -dNOSAFER -sDEVICE=tiff32nc -r150
\end{examplecmd}

\DescribeIF{svg}
The image format generated using \Lprog{pdftocairo}. The line executed by the system is:

\begin{examplecmd}
[user@machine ~:]§ pdftocairo -q -r 150
\end{examplecmd}

\DescribeIF{bmp}
The image format generated using \Lprog{Ghostscript}. The line executed by the system is:

\begin{examplecmd}
[user@machine ~:]§ gs -q -dNOSAFER -sDEVICE=bmp32b -r150
\end{examplecmd}

If the image files exist, they will be rewritten each time you run the script.

\section{How to use}

\subsection{Syntax}

The syntax for \myscript\ is simple, if your use the version provided in your \hologo{TeX} distribution:

\begin{examplecmd}
[user@machine ~:]§  pst2pdf `\small\oarg[type=tt,cf=optcolor,sbc=gray,ac=lightgray]{options} \meta[cf=optcolor]{input file}`
\end{examplecmd}
or
\begin{examplecmd}
[user@machine ~:]§  pst2pdf `\small\meta[cf=optcolor]{input file} \oarg[type=tt,cf=optcolor,sbc=gray,ac=lightgray]{options}`
\end{examplecmd}
If the development version is used:
\begin{examplecmd}
[user@machine ~:]§  perl pst2pdf `\small\oarg[type=tt,cf=optcolor,sbc=gray,ac=lightgray]{options} \meta[cf=optcolor]{input file}`
\end{examplecmd}

The extension valid for \meta[cf=optcolor]{input file} are \Lext{tex} or \Lext{ltx},
relative or absolute paths for files and directories is not supported. If used without
\oarg[type=tt,cf=optcolor,sbc=gray,ac=lightgray]{options} the extracted environments are converted to \Limgform{pdf}
image format and saved in the \texttt{./images} directory using \Lprog{latex}\textcolor{optcolor}{\texttt{\bfseries\guillemotright}}\Lprog{dvips}\textcolor{optcolor}{\texttt{\bfseries\guillemotright}}\Lscript{ps2pdf}
and \LPack{preview} package for process \meta[cf=optcolor]{input file} and \Lprog{pdflatex} for compiler \meta[cf=optcolor]{output file}.

\subsection{Command line interface}
\label{sec:optline}

The script provides a \emph{command line interface} with short \textcolor{optcolor}{\bfseries\texttt{-}}
and long \textcolor{optcolor}{\bfseries\texttt{--}} option, they may be given before
the name of the \meta[cf=optcolor]{input file}, the order of specifying the options is not significant.
Options that accept a \meta{value} require either a blank space \verb*| | or \textcolor{optcolor}{\bfseries\texttt{=}}
between the option and the \meta{value}. Some short options can be bundling.

\DescribeCmd[h]{help}{bolean}{off}
Display a command line help and exit.

\DescribeCmd[l]{log}{bolean}{off}
Write a \LFile{pst2pdf.log} file with all process information.

\DescribeCmd[v]{version}{bolean}{off}
Display the current version (\fileversion) and exit.

\DescribeCmd[V]{verbose}{bolean}{off}
Show verbose information of process in terminal.

\DescribeCmd[d]{dpi}{integer}{150}
Dots per inch for images files. Values are positive integers less than or equal to 2500.

\DescribeCmd[t]{tif}{bolean}{off}
Create a \Lext{tif} images files using \Lprog{Ghostscript}.

\DescribeCmd[b]{bmp}{bolean}{off}
Create a \Lext{bmp} images files using \Lprog{Ghostscript}.

\DescribeCmd[j]{jpg}{bolean}{off}
Create a \Lext{jpg} images files using \Lprog{Ghostscript}.

\DescribeCmd[p]{png}{bolean}{off}
Create a \Lext{png} transparent image files using \Lprog{Ghostscript}.

\DescribeCmd[e]{eps}{bolean}{off}
Create a \Lext{eps} image files using \Lprog{pdftops}.

\DescribeCmd[s]{svg}{bolean}{off}
Create a \Lext{svg} image files using \Lprog{pdftocairo}.

\DescribeCmd[P]{ppm}{bolean}{off}
Create a \Lext{ppm} image files using \Lprog{pdftoppm}.

\DescribeCmd[g]{gray}{bolean}{off}
Create a gray scale for all images using \Lprog{Ghostscript}. The line behind this options is:

\begin{examplecmd}
[user@machine ~:]§ gs -q -dNOSAFER -sDEVICE=pdfwrite -dPDFSETTINGS=/prepress        \
                      -sColorConversionStrategy=Gray -dProcessColorModel=/DeviceGray
\end{examplecmd}

\DescribeCmd[f]{force}{bolean}{off}
Try to capture \lstinline|\psset|\marg[type=tt]{code} to extract.
When using the \cmdopt{force} option the script will try to capture \lstinline|\psset|\marg[type=tt]{code} and
leave it inside the \Lenv{preview} environment, any line that is between \lstinline|\psset|\marg[type=tt]{code} and
\lstinline|\begin{pspicture}| will be captured.

\DescribeCmd[np]{noprew}{bolean}{off}
Create images files without \LPack{preview} package. The \lstinline|\begin{preview}|\ldots \lstinline|\end{preview}|
lines are only used as delimiters for extracting the content \emph{without} using the package \LPack{preview}.
Using this option \emph{\enquote{only}} the extracted environments are processed and not the whole \meta[cf=optcolor]{input file}, sometimes
it is better to use it together with \cmdopt{force}. Alternative name \cmdopt{single}.

\DescribeCmd[m]{margins}{integer}{0}
Set margins in bp for \Lscript{pdfcrop}.

\DescribeCmd[r]{runs}{\textnormal{\ttfamily 1\textbar 2\textbar 3}}{1}
Set the number of times the \meta[type=tt,cf=ForestGreen,ac=lightgray]{compiler} will run
on the \meta[cf=optcolor]{input file} for environment extraction.

\DescribeCmd{myverb}{macro name}{myverb}
Set custom verbatim command \lstinline+\myverb+. Just pass the \meta{macro name}{}
\emph{without} \enquote{\textcolor{optcolor}{\texttt{\bfseries\textbackslash}}}.

\DescribeCmd{ignore}{environment name}{empty}
Add a verbatim environment to internal list.

\DescribeCmd{imgdir}{string}{images}
Set the name of directory for save generated files. Only the \meta{name} of directory
must be passed \emph{without} relative or absolute paths.

\DescribeCmd{srcenv}{bolean}{off}
Create separate files with \emph{\enquote{only code}} for all extracted
environments.

\DescribeCmd{shell}{bolean}{off}
Enable \Lcs{write18}\{SHELL COMMAND\}.

\DescribeCmd[ni]{norun}{bolean}{off}
Execute the script, but do not create image files. This option is designed to
to generate standalone files or used in conjunction with \cmdopt{srcenv} and to
debug the \meta[cf=optcolor]{output file}. Alternative name \cmdopt{noimages}.

\DescribeCmd{nopdf}{bolean}{off}
Don't create a \Lext{pdf} image files.

\DescribeCmd{nocrop}{bolean}{off}
Don't run \Lscript{pdfcrop} in image files.

\DescribeCmd[ns]{nosource}{bolean}{off}
Don't create standalone files.

\DescribeCmd[x]{xetex}{bolean}{off}
Using \Lprog{xelatex} compiler \meta[cf=optcolor]{input file} and \meta[cf=optcolor]{output file}.

\DescribeCmd{luatex}{bolean}{off}
Using \Lprog{dvilualatex}\textcolor{optcolor}{\texttt{\bfseries\guillemotright}}\Lprog{dvips}\textcolor{optcolor}{\texttt{\bfseries\guillemotright}}\Lscript{ps2pdf}
for compiler \meta[cf=optcolor]{input file} and \Lprog{lualatex} for \meta[cf=optcolor]{output file}.

\DescribeCmd{arara}{bolean}{off}
Use \Lprog{arara}\footnote{\url{https://ctan.org/pkg/arara}} tool for compiler \meta[cf=optcolor]{output file}.
This option is designed to full process \meta[cf=optcolor]{output file}, is mutually
exclusive with \cmdopt{latexmk} option.

\DescribeCmd{latexmk}{bolean}{off}
Using \Lscript{latexmk}\footnote{\url{https://www.ctan.org/pkg/latexmk}} for process \meta[cf=optcolor]{output file}.
This option is designed to full process \meta[cf=optcolor]{output file}, is mutually exclusive with \cmdopt{arara}.

\DescribeCmd{zip}{bolean}{off}
Compress the files generated by the script in \texttt{./images} in \Lext{zip} format.
Does not include \meta[cf=optcolor]{output file}.

\DescribeCmd{tar}{bolean}{off}
Compress the files generated by the script in \texttt{./images} in \Lext{tar.gz} format.
Does not include \meta[cf=optcolor]{output file}.

\DescribeCmd{bibtex}{bolean}{off}
Run \Lprog{bibtex} on the \Lext{aux} file (if exists), is mutually
exclusive with \cmdopt{biber} option.

\DescribeCmd{biber}{bolean}{off}
Run \Lprog{biber} on the \Lext{bcf} file (if exists), is mutually
exclusive with \cmdopt{bibtex} option.

\subsection{Example of usage}

An example of usage from command line:

\begin{examplecmd}
[user@machine ~:]§ pst2pdf --luatex -e -p -j --imgdir pics test.ltx
\end{examplecmd}

Create a \texttt{./pics} directory (if it does not exist) with all extracted environments converted to image
formats (\Lext{pdf}, \Lext{eps}, \Lext{png}, \Lext{jpg}) in individual files, an stanalone files (\Lext{ltx}) for all environments extracted,
an output file \LFile{test-pdf.ltx} with all extracted environments converted to \Lcs{includegraphics} and a single file
\LFile{test-fig-all.ltx} with only the extracted environments
using \Lprog{dvilualatex}\textcolor{optcolor}{\texttt{\bfseries\guillemotright}}\Lprog{dvips}\textcolor{optcolor}{\texttt{\bfseries\guillemotright}}\Lscript{ps2pdf}
and \LPack{preview} package for for process \LFile{test.ltx} and
\Lprog{lualatex} for \LFile{test-pdf.ltx}.

\section{Working in another way}

By design, the script generates separate images and files following a predetermined
routine that has already been described in this documentation. Another way to generate
images is as follows:

\begin{enumerate}[font=\small, nosep, noitemsep,leftmargin=*]
\item Execute the script using \cmdopt{norun} to generate \meta{standalone} files, move to \texttt{./images}
and generate \Lext{pdf} files runing:

\begin{examplecmd}
[user@machine~:]§for i in *.tex; do `\meta[type=tt,cf=ForestGreen,ac=lightgray]{compiler} \oarg[type=tt,cf=gray,sbc=optcolor,ac=gray]{options}` $i; done
[user@machine~:]§for i in *.pdf; do pdfcrop `\small\oarg[type=tt,cf=gray,sbc=optcolor,ac=gray]{options}` $i $i; done
\end{examplecmd}

\item Execute the script using \cmdopt{norun}, move to \texttt{./images} \Lext{pdf} file runing:

\begin{examplecmd}
[user@machine~:]§`\meta[type=tt,cf=ForestGreen,ac=lightgray]{compiler} \oarg[type=tt,cf=gray,sbc=optcolor,ac=gray]{options}` test-fig-all.tex
[user@machine~:]§pdfcrop `\oarg[type=tt,cf=gray,sbc=optcolor,ac=gray]{options}` test-fig-all.pdf
\end{examplecmd}

\end{enumerate}

\section{Example files}

The "pst2pdf" documentation provides three example files \LFile{test1.tex}, \LFile{test2.tex}
and \LFile{test3.tex} plus an image file \LFile{tux.jpg} to test and view the script in action.
Copy these files to a directory you have write access to and execute:

\begin{examplecmd}
[user@machine~:]§pst2pdf `\oarg[type=tt,cf=gray,sbc=optcolor,ac=gray]{options}` test1.tex
\end{examplecmd}

%\newpage
\clearpage
\nocite{*}
\printbibliography

\printindex

\end{document}
