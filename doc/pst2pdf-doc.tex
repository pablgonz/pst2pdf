% arara: lualatex: {draft: yes}
% arara: lualatex
% arara: biber
% arara: makeindex
% arara: lualatex
%! arara: lualatex
%! arara: dvips
%! arara: ps2pdf
% arara: clean: { files:[PSTricks.bib,pst2pdf-doc.out, pst2pdf-doc.ps, pst2pdf-doc.toc,pst2pdf-doc.log,pst2pdf-doc.ind,pst2pdf-doc.aux, pst2pdf-doc.bbl, pst2pdf-doc.bcf, pst2pdf-doc.blg,pst2pdf-doc.dvi,pst2pdf-doc.idx,pst2pdf-doc.ilg,pst2pdf-doc.run.xml] }
%\documentclass[11pt,english,BCOR=10mm,DIV=12,bibliography=totoc,parskip=false,headings=small,
    %headinclude=false,footinclude=false,twoside]{pst-doc}
\documentclass[11pt,english,BCOR10mm,DIV12,bibliography=totoc,parskip=false,smallheadings
    headexclude,footexclude,oneside]{pst-doc}
\listfiles
\let\Lfile\LFile
% Identification
\def\myscript{pst2pdf}
\def\fileversion{1.8}
\def\filedate{2020-08-08}
\usepackage{biblatex}
\addbibresource{\jobname.bib}
\lstset{language=PSTricks,basicstyle=\footnotesize\ttfamily}
\def\DVI{\textsc{DVI}}
\def\PDF{\textsc{PDF}}
\def\gs{\textsc{Ghostscript}}
\usepackage{hologo} % LaTeX logos
%%=== new macros ====================================================
\makeatletter
\def\Lscript#1{\texttt{#1}\index{#1@\texttt{#1}}\index{Script!#1@\texttt{#1}}}
\def\Losystem#1{\texttt{#1}\index{#1@\texttt{#1}}\index{Operating system!#1@\texttt{#1}}}
\makeatother
\begin{document}

\title{\texttt{\myscript}}
\subtitle{Running a PSTricks document with pdflatex;\\  \small v\fileversion}
\author{Herbert Voß  \\ Pablo Gonz\'{a}lez Luengo}
\docauthor{}
\date{\filedate}

\maketitle

\begin{abstract}
    \texttt{\myscript} run a \TeX{} \Larga{source}, read all \Lenv{postscript}, \Lenv{pspicture}, \Lenv{psgraph}
    and \Lenv{PSTexample} environments and convert in images pdf,eps,jpg,svg
    or png format (default pdf), extract source code and create new
    file with all environments converted to \Lcs{includegraphics} and runs
    (pdf/Xe/lua)latex.
\end{abstract}

\tableofcontents

\clearpage
\newpage

\section{Introduction}

\PST\ as \PS\ related package uses the programming language \PS\ for internal
calculations. This is an important advantage, because floating point arithmetic is no
problem. Nearly all mathematical calculation can be done when running the \DVI-file
with \gs. However, creating a \PDF\ file in a direct way with \Lprog{pdflatex} is
not possible. \Lprog{pdflatex} cannot understand the \PS\ related stuff.

Instead of running \Lprog{pdflatex} one can use the \Lprog{perl} \emph{script} \Lscript{pst2pdf}, it extracts
all \PST\ related code into single documents with the same preamble as the original
main document.

The \Lscript{pst2pdf} \emph{script} runs document, clips all whitespace around the
image and creates a \Lext{pdf} images of the \PST\ related code.

In a last run which is the \Lprog{pdflatex} the \PST\ code in the
main document is replaced by the created images.

\section{License}

This program is free software; you can redistribute it and/or modify
it under the terms of the \href{https://www.gnu.org/licenses/gpl-3.0.html}{GNU General Public License} as published by
the \href{https://www.fsf.org/}{Free Software Foundation}; either version 3 of the License, or
(at your option) any later version.

This program is distributed in the hope that it will be useful, but
WITHOUT ANY WARRANTY; without even the implied warranty of
MERCHANTABILITY or FITNESS FOR A PARTICULAR PURPOSE. See the \href{https://www.gnu.org/licenses/gpl-3.0.html}{GNU General Public License}
for more details.

\section{Requirements}

For the complete operation of \Lscript{pst2pdf}{} you need to have a modern %
\hologo{TeX} distribution such as \hologo{TeX} Live or \hologo{MiKTeX}, have
a version equal to or greater than \liningnums{5.28} of \Lprog{perl}, a
version equal to or greater than \liningnums{9.24} of \Lprog{Ghostscript}
and have a version equal to or greater than \liningnums{0.52} of %
\Lprog{poppler-utils}. The distribution of \hologo{TeX} Live 2020 for \Losystem{Windows} includes %
\Lscript{pst2pdf} and all requirements, \hologo{MiKTeX} users must install the
appropriate software for full operation.

The script has been tested on \Losystem{Windows} (v10) and %
\Losystem{Linux} (fedora 32) using \Lprog{Ghostscript} %
\liningnums{v9.52}, \Lprog{poppler-utils} \liningnums{v0.84}, \Lprog{perl} \liningnums{v5.30}
and the standard classes offers by \hologo{LaTeX}: \LClass{book},
\LClass{report}, \LClass{article} and \LClass{letter}. The \LPack{preview} and
\LPack{pst-pdf} packages are required to process the \Larga{input file} and if
an \Larga{output file} is generated, the \LPack{graphicx} and \LPack{grfext}
packages will be needed.


\subsection{Preparating file}

The script scan the file for \Lenv{pspicture} and \Lenv{postscript} environments,
which are then taken with its contents from the main file to create stand alone documents
with the same preamble as the main document. The \Lenv{pspicture} environment can be nested,
the \Lenv{postscript} one not! But it can contain an environment \Lenv{pspicture}, but not vice versa.
The \Lenv{postscript} environment should always be used, when there is some code before a \Lenv{pspicture}
environment or for some code which is not inside of a \Lenv{pspicture} environment.

\noindent

\nxLprog{pst2pdf} delete al lines contains \PST\ package before last run, if you need delete other \PST\ code in preamble use:

\begin{verbatim}
%CleanPST
some pstricks code
%CleanPST
\end{verbatim}

\section{Environments support}
\nxLprog{pst2pdf} support fourth environments in default and single way:

\vspace{10pt}
\begin{minipage}[c]{0.25\textwidth}
\begin{verbatim}
\pspicture*
\psset{...}
pstricks code
\endpspicture
\end{verbatim}
\end{minipage}
\begin{minipage}[c]{0.25\textwidth}
\begin{verbatim}
\begin{pspicture}
\psset{...}
pstricks code
\end{pspicture}
\end{verbatim}
\end{minipage}
\begin{minipage}[c]{0.25\textwidth}
\begin{verbatim}
\begin{pspicture*}
\psset{...}
pstricks code
\end{pspicture*}
\end{verbatim}
\end{minipage}
\begin{minipage}[c]{0.25\textwidth}
\begin{verbatim}
\begin{postscript}
\psset{...}
pstricks code
\end{postscript}
\end{verbatim}
\end{minipage}

\vspace{10pt}
Note: When using the default mode, images are created using \Lprog{Ghostscript} and \LPack{preview} package, in this case, it is not necessary
to write \Lenv{psset} into \PST\ environment.

\newpage

\section{Running the script}

\subsection{Default mode}
The general syntax for the \Lprog{perl} \emph{script} is simple:

\begin{BDef}
\nxLprog{perl} \nxLprog{pst2pdf} \Larg{file.tex} \OptArg*{--options}
\end{BDef}

For \TeX Live users:

\begin{BDef}
\nxLprog{pst2pdf} \Larg{file.tex} \OptArg*{--options}
\end{BDef}

In this way \nxLprog{pst2pdf} creates a new file called \Larg{file-pst.tex} and copy all \Lenv{pspicture}
and \Lenv{postscript} environments, then processed and create file-pdf.pdf and file-fig-1.pdf, file-fig-2.pdf, file-fig-\dots.pdf and file-fig-1.tex, file-fig-2.tex, file-fig-\dots.tex for all \Lenv{pspicture} and \Lenv{postscript} using \nxLprog{Ghostscript}.
\subsection{Single mode}
\label{single}
If you do not have \nxLprog{Ghostscript} use the option \Loption{-np,---single} in this mode, the files are processed separately (take a more time to create images files). For example:

\begin{BDef}
\nxLprog{pst2pdf} \Larg{file.tex} \OptArg*{---single}
\end{BDef}

create file-pdf.pdf and file-fig-1.pdf, file-fig-2.pdf, file-fig-\dots.pdf and file-fig-1.tex, file-fig-2.tex, file-fig-\dots.tex for all \Lenv{pspicture} and \Lenv{postscript} environments (see \ref{options}).

\section{Options}
\label{options}
The options listed in Table~\ref{perloptions} refer only to the \emph{script} and not the \LaTeX\ file.

\begin{table}[htp]
\caption{Optional arguments for \nxLprog{pst2pdf}}\label{perloptions}
\begin{tabularx}{\linewidth}{@{} l l >{\ttfamily}l X @{}}\\\toprule
\emph{name} & \emph{values} & \textrm{\emph{default}} & \emph{description}\\\midrule
\Loption{-h,---help}     & boolean & 1        & print help and exit.\\
\Loption{-l,---license}  & boolean & 0        & print license and exit.\\
\Loption{-v,---version}  & boolean & 1     & print version and exit.\\
\Loption{-m,---margins}  & literal & 1   & margins for pdfcrop (in bp).\\
\Loption{-d,---dpi}      & integer & 300      & the dots per inch for a created \Lext{ppm} file.\\
\Loption{-j,---jpg}      & boolean & 0        & creates \Lext{jpg} images (need \Lprog{Ghostscript}).\\
\Loption{-p,---png}      & boolean & 0        & creates \Lext{png} images (need \Lprog{Ghostscript}).\\
\Loption{-e,---eps}      & boolean & 0        & creates \Lext{eps} images (need \Lprog{pdftops}).\\
\Loption{-s,---svg}      & boolean & 0        & creates \Lext{ppm} images (need \Lprog{pdf2svg}).\\
\Loption{-P,---ppm}      & boolean & 0        & creates \Lext{ppm} images (need \Lprog{pdftoppm}).\\
\Loption{-c,---clear}    & boolean & 0        & delete all temporary files.\\
\Loption{-a,---all}      & boolean & 0        & generte all images type and clear.\\
\Loption{-x,---xetex}    & boolean & 0        & using \Lprog{xelatex} instead of \Lprog{latex} for process.\\
\Loption{-np,---single}   & boolean & 0        & create images type (whitout Ghostscript).\\
\Loption{-ni,---noimages} & boolean & 0    		& generate file-pdf.tex, but no images.\\
\Loption{-ns,---nosource} & boolean & 0    		& delte all source for images.\\
\Loption{---imgdir} & literal & images/  & the directory for the created images.\\
\Loption{---ignore} & literal & other   & skip other verbatim environment.\\
\Loption{---Verbose}  & boolean & 1        & for a long \nxLprog{pst2pdf} log.\\
\Loption{---bibtex}   & boolean & 0           & runs \Lprog{bibtex}.\\
\Loption{---bibtex}   & boolean & 0          & runs \Lprog{biber} if a file with extension \Lext{bcf} exists. \\\bottomrule
\end{tabularx}
\end{table}

For Help in command line use:

\begin{BDef}
\nxLprog{pst2pdf} \OptArg*{--help}
\end{BDef}

\section{Other image format}
If you need to create other image formats use \nxLprog{pst2pdf}, move to images dir and use
\nxLprog{mogrify} command (from \Lprog{ImageMagick}), for examples:\\

\begin{BDef}
\nxLprog{mogrify} -format tiff \OptArg*{*.ppm}
\end{BDef}
generate \Lext{tiff} images files.
%\newpage
\clearpage
\nocite{*}
\printbibliography

\printindex
\end{document}
